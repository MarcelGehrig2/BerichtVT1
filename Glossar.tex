\glossary{name={eCAD},description={Electronic Computer-Aided Design: Computerprogramm zum Entwerfen von elektrischen Layouts und PCB}}

\glossary{name={PCB},description={Printed Circuit Board: Elektronische Platine, auf welcher die elektronischen Bauteile aufgelötet werden. Wird auch Platine genannt.}}

\glossary{name={BBB},description={BeagleBone Black: Kostengünstiger Einplatinen-Computer von Texas Instruments}}

\glossary{name={BBB-Derivat},description={BeagleBone-Derivat: Das Produkt, das in dieser Bachelorarbeit in Zusammenarbeit mit Variosystems hergestellt wurde. Es hat den selben Funktionsumfang wie der BBB}}

\glossary{name={BeagleBone Black},description={Siehe BBB}}

\glossary{name={PHY},description={PHYsical Layer: Spezieller integrierter Schaltkreis der die modulierten analogen Daten des LAN-Anschlusses in digitale Daten wandelt und umgekehrt. Dieses Bauteil wird zusammen mit einer RJ45-Buchse und einem MAC für eine Ethernet-Schnittstelle benötigt}}

\glossary{name={MAC},description={Media Access Control: Dieser Controller wird zusammen mit einer RJ45-Buchse und einem PHY für eine Ethernet-Schnittstelle benötigt }}

\glossary{name={HDMI},description={High Definition Multimedia Interface: Eine Schnittstelle für die volldigitale Übertragung von Audio- und Videodaten}}

\glossary{name={eMMC},description={Embedded Multimedia Card: Massenspeicher mit Flashspeicher, MMC-Interface und Controller. Ersetzt im BBB die Festplatte}}

\glossary{name={awk},description={Skriptsprache zur Bearbeitung und Auswertung strukturierter Textdaten, beispielsweise CSV-Dateien}}

\glossary{name={Designator},description={Eindeutige Bezeichnung aus einem Buchstaben und einer Zahl für ein elektrisches Bauteil in einem Schema oder PCB-Layout}}

\glossary{name={Cape},description={Erweiterung speziell für den BBB}}

\glossary{name={WLAN},description={Kabelloser Standard für LAN}}

\glossary{name={BLE},description={Bluetooth Low Energy: Standard für eine Funktechnik, die mit sehr geringen Stromverbrauch Geräte in einer Umgebung von 10 Meter vernetzen kann}}

\glossary{name={LCD},description={Liquid Cristal Display: Flachbildschirm}}

\glossary{name={SDIO},description={Secure Digital Input Output: Vielseitiger Datenbus, welcher unter Anderem für SD-Karten verwendet wird}}

\glossary{name={I$^2$C},description={Inter-Integrated Circuit: serieller Datenbus, welcher für die Kommunikation zwischen Bauteilen auf einem PCB verwendet wird}}

\glossary{name={EEPROM},description={Electrically Erasable Programmable Read-Only Memory: nichtflüchtiger, elektronischer Speicher für kleine Datenmengen}}

\glossary{name={RAM},description={Random-Access Memory: wird von Prozessoren als Arbeitsspeicher benötigt}}

\glossary{name={Python},description={Universelle Programmiersprache}}

\glossary{name={SMT},description={Surface Mounted Technology: Oberflächenmontage. Die elektrischen Bauteile werden nur auf der Oberfläche des PCB gelötet}}

\glossary{name={THT},description={Through Hole Technology: Durchsteckmontage. Die Bauelemente haben Drähte als Anschluss, die durch das PCB gesteckt werden}}

\glossary{name={McASP},description={Multichannel Audio Serial Port: Datenbus für Audiodaten}}

\glossary{name={Footprint},description={Die Umrisse von Lötflächen von elektrischen Bauelementen auf einer Leiterplatte}}





\glossary{name={USB},description={Universal Serial Bus: Serielles Bussystem zur Verbindung eines Computers mit externen Geräten}}

\glossary{name={LAN},description={Local Area Network: Netzwerk mit einer Ausdehnung, von i.d.R. mehreren Räumen}}

\glossary{name={SPI},description={Serial Peripheral Interface: Bus-System nach dem Master-Slave-Prinzip zur Verbindung von digitalen Schaltungen}}

\glossary{name={GPIO},description={General Purpose Input/Output: Schnittstelle die die meisten Mikrocontroller besitzen um mit externen Geräten zu kommunizieren}}

\glossary{name={RS232},description={Standard für eine serielle Schnittstelle mit definiertem Spannungspegel}}

\glossary{name={UART},description={Standard für eine serielle Schnittstelle}}





\renewcommand{\glossaryname}{Glossar}
\printglossary
