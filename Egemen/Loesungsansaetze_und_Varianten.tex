\chapter{Lösungsansätze und Varianten}


\section{Strategie}
% möglichst einfach, sicher funktioniern



\section{Grundsätzlicher Aufbau}
Für den den grundsätzlichen Aufbau des ganzen Systems gab es zwei Möglichkeiten.

\subsection{Variante 1: Eine grosse Platine}
Das ganze System, also den BBB und die Bauteile für die Zusatzfunktionen, werden auf einer einer einzigen Platine platziert.

Vorteile:
\begin{itemize}
\item Alles ist in einem einzigen, geschlossenen Projekt. Dies hat den Vorteil, dass nur ein grosses PCB bestellt und bestückt werden muss. Dies ist günstiger und einfacher als zwei kleine. So muss zum Beispiel die Pick an Place Maschine nur einmal eingerichtet werden.
\item Flexible Form des PCB. Je nach dem wie die Bauteile platziert werden, kann die Form des PCB beeinflusst werden.
\item Die relativ teuren Steckverbindungen zwischen den Platinen fallen weg.
\end{itemize}

\subsection{Variante 2: Die Zusatzfunktionen als Cape}
Das Konzept des originalen BBB wird beibehalten. Es werden zwei PCBs hergestellt. Das erste PCB ist von der Funktion her identisch, wie das originale BBB. Auch die Steckverbindungen für die Capes werden beibehalten. Zusätzlich wird noch eine zweite Platine entwickelt, die auch mit dem originalen BBB kompatibel ist und alle Bauteile für die Zusatzfunktionen enthält.

Vorteile:
\begin{itemize}
\item Entwicklungsfreundlicher. 
\item asd
%TODO Egemen
%\item Kompatibel mit dem orginal BBB
%\item Kompakter
%\item leichter durch Capes erweiterbar
\end{itemize}

%TODO Egemen
\section{Wahl des Betriebssystems}
Der BeagleBone unterstützt mehrere auf Linux basierte Betriebssysteme, die in ihren Grundfunktionen auf die gleiche Art und Weise gleich Funktionieren aber dennoch Unterschiede in Bezug auf Bedienung, Funktionsvielfalt,Support oder Kompatibilität vorweisen.
Aber auch die Wahl des Betriebsystems, auf dem die eigentliche Entwicklung der Software stattfindet, ist wichtig. 

\subsection{Betriebssystem für das BeagleBone}
Das Orginal BBB wird mit dem Betriebssystem, kurtz BS, Debian.Neben dem BS Debian werden auch andere BS unterstützt, wie beispielsweise Ubuntu, Android oder Angstrom. Da alle auf Linux basieren, gleichen sie sich in den Grundfunktionen die der BBB bietet und zu Verfügung stellt. Die unterscheide liegen in der Bedienung der Betriebssysteme und die Bereitstellung und Unterstützung von Funktionen und Treiben. Da Debian eine hohe Kompatibilität zu vielen Cape's und somit eine grosse Funktionsvielfalt durch das BS an sich und durch andere Entwickler bietet, wurde bei dieser Arbeit auf Debian gesetzt. 
Zusätzlich dem benutztem BS ist auch die Wahl des Kernels wichtig.der BBB wird mit dem Kernel in der Version 3.8.13 ausgeliefert. Neuere Versionen des Kernels sind zwar verfügbar, jedoch wurden bei diesen nicht alle Funktionen und Treiber des BBB's auf Funktion und Stabilität getestet. Die Version 3.8.13 gilt als sehr Stabil und bietet eine Kompatibilität zu vielen Cape's, weshalb die Wahl auf diese Version viel.

\subsection{Betriebssystem für die Entwicklungumgebung}
Auch hier bietet sich eine Vielfalt von Betriebssystemen an. Grundsätzlich ist die erste Entscheidung die man treffen muss ob ein Linux oder Windows gewählt werden sollte.
In diesem Fall bietet ein Linux Betriebssystem Vorteile gegenüber Windows, da das BeaglBone ebenfalls auf Linux setzt. Dadurch ist eine hohe Kompatibilität und Übereinstimmung in der Bedienung und Ausführung von Funktionen von vornherein sichergestellt ohne weiter Treiber oder Anwendungen zu installieren oder andere Einstellungen vorzunehmen.
Daher wurde für die Entwicklungsumgebung auf das Betriebssystem Ubuntu 14.04 LTS (Long Term Support) gesetzt. Diese Version von Ubuntu bietet fast die selbe Vielfalt an unterstützter Software wie Windows womit ein Umstieg auf dieses Betriebssystem leichter fällt als bei anderen Linux-Systemen.
Bei der Entwicklung wurde nicht komplett auf Windows verzichtet, da die Entwicklungsumgebung für das Bluetooth Modul, von der Hersteller Seite aus, vermehrt auf die Windowsumgebung angepasst ist.