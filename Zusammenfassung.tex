
\chapter*{Kurzfassung}
In dieser Arbeit wird ein neuer Sequencer für EEROS vorgestellt.

EEROS ist ein echtzeitfähiges Roboter-Framework, welches an der Interstaatlichen Hochschule für Technik Buchs (NTB) entwickelt wurde, und immer noch weiter entwickelt wird.
Das Framework ist Open Source und frei erhältlich.

EEROS besteht aus den drei Hauptteilen Sequencer, SafetySystem und ControlSystem.
Der Sequencer ist dabei der Teil der Software, der den Ablauf des Roboters steuert.

In dieser Arbeit wurden zuerst der bestehende Sequencer analysiert, um die Schwächen und Stärken zu ermitteln.
Im Anschluss wurden die Anforderungen für den neuen Sequencer definiert.

Es wurde ein Set von Testfällen ausgearbeitet, welche alle Anforderungen abdecken.
Die Testfälle wurden genutzt um ein Sequencer zu entwickeln, der alle Testfälle abdeckt.

Bei der Entwicklung des Sequencer wurde auf zwei Hauptfokuspunkte geachtet.
Zum einen sollte er möglichst flexibel sein, damit EEROS auch weiterhin bei einer Vielzahl von sehr unterschiedlichen Robotern eingesetzt werden kann.
Der zweite Fokuspunkt war er möglichst einfache Aufbaue einer Sequenz.
Eine fertige Sequenz sollte so einfach sein, dass auch ein Nicht-Fachmann den Ablauf versteht und geringfügige Änderungen machen kann.