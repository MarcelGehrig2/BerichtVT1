
\chapter*{Zusammenfassung}
Im Auftrag des Industriepartner Variosystems wurde ein kostengünstiger, auf dem BeagleBone Black basierter  Platinencomputer entwickelt. Der BeagleBone Black ist ein vollständiger Computer mit einem Ubuntu Linux als Betriebssystem. Im verlaufe dieser Arbeit wurden insgesamt 5 Exemplare hergestellt, die alle in der Variosystems bestückt wurden. Mit einem Cape, eine aufsteckbare Platine für den Computer, wurde der Computer mit WLAN, Bluetooth Low-Energy, GSM/GPRS und einem Touchscreen ergänzt. Dieses Cape ist nicht nur mit dem von uns gebauten BeagleBone Black Derivat kompatibel, sondern auch mit dem kommerziell erhältlichen, originalen BeagleBone Black. Die Kombination des BeagleBone Black und dem Cape wird im Folgenden Communication-Bone, oder kurz ComBone genant. Der Name ist eine Wortkombination des englischen Wortes "Communication" für die Kommunikationsfähigkeit des Capes über verschiedene Kanäle, sowie dem Wort "Bone", welches bereits im Namen des originalen BeagleBone Black genutzt wird.

%TODO Egemen
%vorschlag: anstatt dauernd BeagleBone Black die abkürzung BBB benutzen
%Der BeagleBone Black, im weiterm Text als BBB bezeichnet,...
%
%Der BBB ist ein vollständiger Computer für Linux basierte Betriebssysteme.
%Standard mässig wird es mit dem Betriebssystem Debian ausgeliefert, welches für die BA ebenfalls benutzt wird.
%

Bei der Entwicklung der Hard- und Software ist darauf geachtet worden, dass die einzelnen Funktionen möglichst modular sind. Wenn bestimmte Funktionen nicht benötigt werden, wie zum Beispiel der HDMI Anschluss des BeagleBone oder das WLAN-Funktion des Capes, können die entsprechenden Bauteile bei der Produktion einfach nicht bestückt werden. Dies kann, besonders bei grösseren Stückzahlen, viel Geld sparen. Des weiteren können auch einige Module, beziehungsweise Funktionen, einfach kopiert und in anderen Projekten verwendet werden.

Ein möglicher Einsatzbereich dieses Computer mit dem Cape ist die Verbindung von einem Gerät, wie etwa ein Sensor oder ein abgelegener Stromgenerator, mit dem Internet. Da sich der ComBone mit einer LAN-Verbindung, mit WLAN und über das mobile GSM Netz, wie es auch ein Mobiltelefon verwendet, ins Internet einwählen. Dies macht den ComBone ein hochflexibles Gerät, welches diverse Einsatzmöglichkeiten hat.





\section*{Abstract}
%\thispagestyle{empty}
